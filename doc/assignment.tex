\documentclass[10pt,english]{article}
%\documentclass[10pt,italian]{article} % TO WRITE IN ITALIAN, UNCOMMENT THIS LINE AND COMMENT THE ABOVE ONE
\usepackage[utf8]{inputenc} % opzione per caratteri ISO-8859-1, CONSENTE L'USO DELLE ACCENTATE
\usepackage{hyperref}

% MARGINI LARGHI
\textwidth 6.3 in % Width of text line.
    \textheight 9.2 in
    \oddsidemargin 0 in      %   Left margin on odd-numbered pages.
    \evensidemargin 0 in      %   Left margin on even-numbered pages.
    \topmargin 0.2 in
    \headheight 0 in       %   Width of marginal notes.
    \headsep 0 in
    \topskip 0 in
    
\title{\vspace{-70pt}Assignment 03 -- ``Prolog''}
\author{Jacopo Riciputi, matr: 885773, email: {\url{jacopo.riciputi@studio.unibo.it}}\\ repo: {\url{https://github.com/Jacopo47/pps2017-asg03-prolog}} }
\date{\today}


\begin{document}

\maketitle
\vspace{-30pt}

\subsection{Descrizione}

Per portare a termine l'assignment ho optato per la crezione di un semplice gioco in stile \textbf{Akinator}. \\
L'applicazione non fa altro che chiedera al giocatore di pensare a un insieme limitato di personaggi della serie tv Breaking Bad per poi porre delle domande all'utilizzatore fino a capire in base alle risposte il personaggio selezionato. \\
Il gioco dopo aver dato qualche secondo all'utente per individuare il personaggio di riferimento si avvia in una serie di domande. 
Ad ogni risposta ottenuta cerca all'interno dei propri predicati quelli che coincidono con le risposte date, fino ad arrivare al punto di selezionare l'unico personaggio che corrisponde al profilo esatto tracciato dalle risposte del giocatore.


\subsection{Tecniche usate}
Per completare l'assignment avevo la necessità di un ambiente che mi permettesse di gestire comodamente le interazioni con l'utente.
Per questo motivo ho optato per Scala, grazie al quale è risultato semplice creare le parti di modello e di controllo. \\
La scelta è stata guidata verso Scala anche per la possibilità di poter sfruttare il motore \textbf{tuProlog}, l'intera applicazione infatti si basa su di esso. \\

L'utilizzo di Prolog si può riassumere in questi punti: 
  \begin{itemize}
   \item All'avvio del programma infatti vengono caricati i personaggi definiti staticamente all'interno di un motore tuProlog come \textit{Theory}.
   \item A ogni risposta dell'utente l'applicazione rimanda al motore creato la ricerca dei personaggi che corrispondo al profilo delineato. 
   \item Una volta ottenuti viene creato un iteratore che permette un attraversamento facilitato dei risultati. 
  \end{itemize}



\subsection{Autovalutazione}
L'utilizzo di Prolog all'interno di un progetto Java/Scala è stata una bella scoperta, le librerie \textit{tuProlog} forniscono un'ottima integrazione e davvero intuitiva. \\
Devo però ammettere che, probabilmente per colpa della poca esperienza, fatico a vedere il Prolog come possibile soluzione a richieste di miei eventuali progetti. 
\\
Anche se, come indicato precedentemente, la poca esperienza con uno stile di programmazione completamente nuovo e mai visto prima mi porta verso questo pensiero. \\
Nonostante abbia capito i suoi concetti basilari avrei la necessità di vederlo in un vero dominio applicativo, così da riuscire a capire le sue piene capacità. 


 
\end{document}
    
    
